\documentclass{article}
\usepackage[utf8]{inputenc}
\usepackage{geometry}
\geometry{margin=1in}
\usepackage{hyperref}

\title{iGame-center Overview}
\author{Mason}
\date{\today}

\begin{document}

\maketitle

\section*{Purpose}
iGame-center is a browser-based game platform that tracks player progress, scores, and achievements. It provides a centralized hub for players to play games, view stats, and compete globally.

\section*{Core Components}
\begin{itemize}
    \item \textbf{index.html} -- Main game interface where players play the game.
    \item \textbf{scoreboard.html} -- Displays global scores and player rankings.
    \item \textbf{gamercard.html} -- Shows detailed player stats and achievements.
    \item \textbf{src/core/} -- Contains all frontend code (HTML, JS, CSS) that runs in the browser.
\end{itemize}

\section*{Backend / Data Handling}
\begin{itemize}
    \item \textbf{GraphQL API} (via server.js and graphql/) handles storing and querying scores and achievements. It allows the frontend to retrieve exactly the data it needs.
    \item \textbf{In-memory storage} (currently in graphql/data.js) temporarily stores scores and achievements; can be upgraded to a database later.
\end{itemize}

\section*{Features}
\begin{itemize}
    \item Global Score Tracking -- Players' scores are tracked and displayed on a leaderboard.
    \item Gamercard System -- Personalized player stats.
    \item Achievements -- Planned for future updates (2.1+).
    \item Update Logging -- Optional documentation using \texttt{.tex} files for changelogs.
\end{itemize}

\section*{Technology Stack}
\begin{itemize}
    \item Frontend: HTML, JavaScript
    \item Backend: Node.js + Apollo Server
    \item API: GraphQL
    \item Storage: In-memory JS arrays (future: database)
    \item Optional: TeX (\texttt{.tex} files) for update overviews or documentation
\end{itemize}

\section*{Workflow Overview}
\begin{enumerate}
    \item Player plays game on \textbf{index.html}.
    \item Score is sent via GraphQL mutation to backend.
    \item Backend stores the score.
    \item \textbf{scoreboard.html} queries backend via GraphQL to show updated leaderboard.
    \item Gamercard queries backend to display stats and achievements.
    \item Updates can be documented in \texttt{.tex} files for professional changelogs.
\end{enumerate}

\end{document}
\end